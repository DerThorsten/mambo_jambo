% !TEX root = ../../main.tex

\section{Graph API's}\label{sec:graph_apis}

We strongly belief is is beneficial to use an existing API's for
graphs instead of inventing an own graph API for Vigra.
Coming up with a new API is not straight forward
since one might to think of any future use case this API.
In addition, users became accustomed with existing API's as 
the LEMON or Boost graph API.
Existing algorithms of a particular API can be reused if we stick
to that API.




\subsection{LEMON Graph API's}\label{sec:lemon_graph_apis}
    LEMON \citet{ software_lemon} 
    stand for  ``Library for Efficient Modeling and Optimization in Networks.''.
    It is an open source C++ library with algorithms and data structures 
    related to directed and undirected graphs.
    The extensive usage of templates make this library very flexible.
    While lemon provides a huge set of graph algorithms,
    we are mostly interested in the graph API itself.
    In the following we will give a brief overview of lemons graph 
    API and the related concepts.
    Explaining the complete lemon graph API in detail
    is beyond the scope of this thesis.
    Interested readers are referred to \citet{software_lemon}.
    We will only discuss the API for undirected graphs since any
    graph algorithm we implemented within this thesis
    will work on undirected graphs.

\subsubsection{Graph Items}
    Any undirected graph class fulfilling the lemon API needs to define 
    the following \emph{descriptor} types to represent the graph items.
    \begin{compactitem}
    \item \lstinline{Graph::Edge}
    \item \lstinline{Graph::Arc}
    \item \lstinline{Graph::Node}
    \end{compactitem}
    These \emph{descriptor} should be cheap types which can be copied
    and passed with almost no overhead.
    In addition, descriptor has an unique id
    \footnote{ unique id w.r.t. the item type. 
    Therefore  multiple  nodes cannot have the same id.
    The same holds line for edges and arcs.
    But there might be a node and edge which have the same id}.
    These id's can be accessed via \lstinline{Graph::id(Node)}, \lstinline{Graph::id(Edge)} and \lstinline{Graph::id(Arc)}.
    These id's can not only be dense but also sparse, which is very
    important for an efficient handling of grid graph edge ids.



\subsubsection{Iterators}

    Within lemon a very convenient mechanism is used to iterate over
    nodes, edges and arcs.
    A special constant \lstinline{INVALID} is used to determine if 
    an iterator reached the end.

    \begin{minipage}{\textwidth}\vspace{-0.75cm}\begin{lstlisting}[language=c++]
    // iterate over nodes
    for(Graph::NodeIt v(g); v!= lemon::INVALID; ++v){/*...*/}

    // iterate over edges
    for(Graph::EdgeIt e(g); e!= lemon::INVALID; ++e){/*...*/}

    // iterate over arcs
    for(Graph::ArcIt a(g); a!= lemon::INVALID; ++a){/*...*/}

    // use arcs to iterate over neighbor nodes
    for(Graph::OutArcIt a(g,n); a!= lemon::INVALID; ++a){
        const Node neighborNode = g.target(a);
        /*...*/
    }
    \end{lstlisting}\end{minipage}\vspace{0.5cm}

    Any iterator is convertible to the corresponding item which
    is iterated without using \lstinline{operator*()}.

\subsubsection{Map Concept}

    In lemon, graph classes store only the structure of the graph itself.
    All addition data for nodes, edges and arcs is stored 
    in \emph{maps}. 
    Any graph has default implementations for graph maps which
    can be accessed in the following way.

    \begin{minipage}{\textwidth}\vspace{-0.75cm}\begin{lstlisting}[language=c++]
    // edge map (for data as edge weights)
    Graph::EdgeMap<float> edgeMap(g); 

    // read and write data 
    for(Graph::EdgeIt e(g); e!= lemon::INVALID; ++e){
        // read
        const float val = edgeMap[*e];
        // write
        edgeMap[*e] = std::exp(-1.0*a);
    }

    // node map (for node related data as node labelings )
    Graph::NodeMap<usigned int> nodeMap(g);
    for(Graph::NodeIt v(g); v!= lemon::INVALID; ++v){
        // read
        const unsigned int val = nodeMap[*e];
        // write
        nodeMap[*e] = val+1;
    }
    \end{lstlisting}\end{minipage}\vspace{0.5cm}


    Custom maps can be implemented very easy:

    \begin{minipage}{\textwidth}\vspace{-0.75cm}\begin{lstlisting}[language=c++]
    template<class Graph>
    class ImplicitEdgeMap {
    public:
        typedef typename Graph::Edge Key;
        typedef double Value;
        Value operator[](Key edge) const { 
            Value a;
            /*...*/
            return a;
        }
    };
    \end{lstlisting}\end{minipage}\vspace{0.5cm}

\subsection{BOOST Graph API's}\label{sec:boost_graph_apis}
The Boost Graph Library (BGL)  \cite{software_bgl} is set of data structures and 
algorithms for graph related computations.
Since all algorithms are implemented within the LEMON graph interface, 
the BGL graph API will only be described briefly.
The graph API defined in the BGL is a collection of
free functions and graph traits which can be specialized for
any graph.




\subsection{VIGRA Graph API}

While the grid graph is implemented within the BGL and LEMON graph API,\
all algorithms implemented within this thesis will
use the LEMON graph API instead of the BGL 
for the following reason.
\todo{write reasons...}

