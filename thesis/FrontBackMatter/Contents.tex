% !TEX root = ../main.tex
% Table of Contents - List of Tables/Figures/Listings and Acronyms

\refstepcounter{dummy}

\pdfbookmark[1]{\contentsname}{tableofcontents} % Bookmark name visible in a PDF viewer

\setcounter{tocdepth}{2} % Depth of sections to include in the table of contents - currently up to subsections

\setcounter{secnumdepth}{3} % Depth of sections to number in the text itself - currently up to subsubsections

\manualmark
\markboth{\spacedlowsmallcaps{\contentsname}}{\spacedlowsmallcaps{\contentsname}}
\tableofcontents 
\automark[section]{chapter}
\renewcommand{\chaptermark}[1]{\markboth{\spacedlowsmallcaps{#1}}{\spacedlowsmallcaps{#1}}}
\renewcommand{\sectionmark}[1]{\markright{\thesection\enspace\spacedlowsmallcaps{#1}}}

\clearpage

\begingroup 
\let\clearpage\relax
\let\cleardoublepage\relax
\let\cleardoublepage\relax

%----------------------------------------------------------------------------------------
%	List of Figures
%----------------------------------------------------------------------------------------

\refstepcounter{dummy}
%\addcontentsline{toc}{chapter}{\listfigurename} % Uncomment if you would like the list of figures to appear in the table of contents
\pdfbookmark[1]{\listfigurename}{lof} % Bookmark name visible in a PDF viewer

\listoffigures

\vspace*{8ex}
\newpage

%----------------------------------------------------------------------------------------
%	List of Tables
%----------------------------------------------------------------------------------------

\refstepcounter{dummy}
%\addcontentsline{toc}{chapter}{\listtablename} % Uncomment if you would like the list of tables to appear in the table of contents
\pdfbookmark[1]{\listtablename}{lot} % Bookmark name visible in a PDF viewer

\listoftables
        
\vspace*{8ex}
\newpage
    
%----------------------------------------------------------------------------------------
%	List of Listings
%---------------------------------------------------------------------------------------- 

\refstepcounter{dummy}
%\addcontentsline{toc}{chapter}{\lstlistlistingname} % Uncomment if you would like the list of listings to appear in the table of contents
\pdfbookmark[1]{\lstlistlistingname}{lol} % Bookmark name visible in a PDF viewer

\lstlistoflistings 

\vspace*{8ex}
\newpage
       
%----------------------------------------------------------------------------------------
%	Acronyms
%----------------------------------------------------------------------------------------

\refstepcounter{dummy}
%\addcontentsline{toc}{chapter}{Acronyms} % Uncomment if you would like the acronyms to appear in the table of contents
\pdfbookmark[1]{Acronyms}{acronyms} % Bookmark name visible in a PDF viewer

\markboth{\spacedlowsmallcaps{Acronyms}}{\spacedlowsmallcaps{Acronyms}}

\chapter*{Acronyms}

\begin{acronym}[UML]
\acro{DRY}{Don't Repeat Yourself}
\acro{API}{Application Programming Interface}
\acro{UML}{Unified Modeling Language}
\end{acronym}  
                   
\endgroup

\cleardoublepage



%----------------------------------------------------------------------------------------
%   Notation
%----------------------------------------------------------------------------------------
\refstepcounter{dummy}
%\addcontentsline{toc}{chapter}{Notation} % Uncomment if you would like the notation to appear in the table of contents
\pdfbookmark[1]{Notation}{notation} % Bookmark name visible in a PDF viewer

\markboth{\spacedlowsmallcaps{Notation}}{\spacedlowsmallcaps{Notation}}

\chapter*{Notation}

Within this thesis we will use the following Notation:

\begin{compactitem}
    \item $w$ is used for edge weights.
    \item $F^E$ will be used for an edge feature vectors and $F^E$ will be used
        for node feature vectors.
    \item The superscript ``$+$'' (e.g. $w^+,\mathfrak{D}^+$) will be used to indicate 
        that high values of the superscripted symbol stand for
        measurements which themselves indicate that the two nodes of the edge should stay in separated
        connected components. And low values indicates that the two regions should be merged.
        The superscript ``-'' indicates the opposite.

    \item Capital fractal letters as $\mathfrak{D,W}$ will be used as
        functions transferring feature vector(s) to a single scalar, 
        therefore $\mathbb{R}_0^N,\ldots,\mathbb{R}_k^N, \rightarrow \mathbb{R} $ .
        $\mathfrak{W}$ will indicate a function transferring an edge feature
        vector to an edge weight, $\mathfrak{D}$ will indicate a function transferring 
        node features distances to edge weights.
\end{compactitem}





%UTC\nomenclature{UTC}{Coordinated Universal Time} 
%is 3 hours behind ADT\nomenclature{ADT}{
%Atlantic Daylight Time} and 10 hours ahead of
%EST\nomenclature{EST}{Eastern Standard Time}.
%\printnomenclature