% !TEX root = ../main.tex
\chapter{Region Adjacency Graph}\label{ch:regio_adjacency_graph}

    
  \begin{MyDef}[Id-Graph]
    A graph $G = (U, V )$ is called \emph{id-graph} if there is
    bijective function  $\mathcal{N}_{\text{id}}: U \rightarrow \mathbb{N}_0 $ 
    mapping any node $\in U$ to a unique non negative id.
  \end{MyDef}
    

  \begin{MyDef}[Region Adjacency Graph]
    Given a base graph $\baseGraph = (\baseNodes, \baseEdges)$ and 
    a node labeling $\Labels_{i}^{base} \in  \mathbb{N}_0 \quad \forall \quad u_i \in \baseNodes$:

    A region adjacency graph  is defined by an id-graph  $\ragGraph = (\ragNodes, \ragEdges)$,
    a surjective function $ \mathcal{N}_{\text{RAG}}^{\uparrow} : \baseNodes \rightarrow \ragNodes$ which maps
    any node from the $\baseGraph$ to exactly one node in $\ragGraph$ s.t:
    \[
        \quad s.t. \quad
        \mathcal{N}_{\text{id}}^{\text{RAG}} 
        \left(
            \mathcal{N}_{\text{RAG}}^{\uparrow}
            \left(
                u_i^{\text{base}}
            \right)
        \right) 
        = \Labels_i^{base}
        \quad \forall u_i^{\text{base}} \quad \in \baseNodes
    \]

    and a function $\mathcal{E}_{\text{RAG}}^{\downarrow} \quad : \quad \ragEdges \rightarrow \{  \baseEdges \} $
    s.t:
    \begin{align*}
        |\mathcal{E}_{\text{RAG}}^{\downarrow}(u_i^{\text{RAG}})| >=1 
            \quad \forall u_i^{\text{RAG}} \quad \in \ragNodes  \\
        %
        \sum _{u_i^{\text{RAG}} \quad \in \ragNodes } 
            |\mathcal{E}_{\text{RAG}}^{\downarrow}(u_i^{\text{RAG}})| = |\baseEdges|
    \end{align*}

  \end{MyDef}



To get a region adjacency graph (RAG), a labeled graph is needed.
The labeled graph will hereinafter be referred to as \emph{base graph} 
$G_{\text{base}}=(V_{\text{base}},E_{\text{base}})$
of the RAG $G_{\text{RAG}}=(V_{\text{RAG}},E_{\text{RAG}})$

Each node $u_i \in V_{\text{base}} $ is assigned a 
label $\Labels_{i} \in  \mathbb{N}_0$.
For each unique label in $\Labels$ a node is added to the RAG, 
and the id of that node is the label itself.

For each $e_{ij} \in E_{\text{base}}$ where $L_i \neq L_j$ an edge is added to the RAG,
if and if only no edge between $u_{\Labels_i}$ and $u_{\Labels_j}$ has been in 



This allows an easy mapping between $V_{\text{base}}$ and $V_{\text{RAG}}$, 
since $\Labels$  can be used as \emph{forward mapping} from $V_{\text{base}}$ to $V_{\text{RAG}}$.


