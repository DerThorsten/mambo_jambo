% !TEX root = ../main.tex
% Abstract
\pdfbookmark[1]{Abstract}{Abstract} % Bookmark name visible in a PDF viewer

\begingroup
\let\clearpage\relax
\let\cleardoublepage\relax
\let\cleardoublepage\relax

\chapter*{Abstract} % Abstract name

Recently, unsupervised image segmentation has become increasingly
popular.
Starting from a superpixel segmentation, an edge-weighted region
adjacency graph is constructed. Amongst all
segmentations of the graph,
the one which best conforms to the given image
evidence, as measured by the sum of cut edge weights,
is chosen.

Since this problem is NP-hard, we propose 
a new approximate solver based on the move-making paradigm:
first, the graph is recursively partitioned into
small regions (cut phase).
%
Then, for any two adjacent 
regions, we consider alternative cuts of these two regions
defining possible moves (glue \& cut phase).
%
For planar problems, the optimal move can be found, whereas
for non-planar problems, efficient approximations exist. 

We evaluate our algorithm on published and
new benchmark datasets, which we make available here.
%
The proposed algorithm finds segmentations that,
as measured by a loss function, are as close to
the ground-truth as the global optimum found by exact solvers.
%
It does so significantly faster then existing approximate methods,
which is important for large-scale problems.

Furthermore we provide a library for graph based image analysis implemented
in C++ within the  VIGRA library, with Python wrappers
for easy usage.
The graph library consists of several graph classes, all 
sharing a common API and a set of generic algorithm working
on all implemented graphs.
The main aspects of the graph framework is usability and extendability.


\endgroup			

\vfill