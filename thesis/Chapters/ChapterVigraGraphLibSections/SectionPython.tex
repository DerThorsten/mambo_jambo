% !TEX root = ../../main.tex
\section{Python}\label{sec:graph_lib_python}

\begin{flushright}{\slshape    
(1) Beautiful is better than ugly. \\ \label{cit:line_a}
(2) Explicit is better than implicit. \\ \label{cit:line_b}
(3) Simple is better than complex. \\
(4) Complex is better than complicated. \\
(5) Flat is better than nested. \\
(6) Sparse is better than dense. \\
(7) Readability counts. \\
(8) Special cases aren't special enough to break the rules. \\
(9) Although practicality beats purity. \\
(10) Errors should never pass silently. \\
(11) Unless explicitly silenced. \\
(12) In the face of ambiguity, refuse the temptation to guess. \\
(13) There should be one-- and preferably only one --obvious way to do it. \\
(14) Although that way may not be obvious at first unless you're Dutch. \\
(15) Now is better than never. \\
(16) Although never is often better than *right* now. \\
(17) If the implementation is hard to explain, it's a bad idea. \\
(18) If the implementation is easy to explain, it may be a good idea. \\
(19) Namespaces are one honking great idea -- let's do more of those! } \\ \medskip
--- The Zen of Python
\end{flushright}
\captionof{figure}{ 
    The Zen of Python
}\label{fig:zen_of_python}


\subsection{Graph Maps}

On the python side, we want node-maps, edge-maps and arc-maps to be stored 
as numpy arrays for several reasons.
Numpy arrays are the standard for storing multidimensional data in python.
The fast C implementation and the highly vectorized API of numpy makes it very easy to write 
fast python code within a few lines.
Virtually any python user will be familiar with the numpy API and therefore it 
seems to be natural to store graph maps within numpy arrays.

In addition VIGRA provides an mechanism to pass numpy arrays to C++.
Therefore no new mechanism needs to be implemented to transfer graph
maps from python to C++.
This will not only simplify writing extension for the new VIGRA graph API,
but also it will reduce the glue code since we can use well tested existing
solutions.

New algorithms might be implemented in pure python with a mix of
existing numpy functions and new functions provided within VIGRA's graph API.
As a proof of concept we implemented ??? in pure python with VIGRA's
fresh graph API in \cref{???}.


\subsubsection{Intrinsic Graph Shape}

Each graph has an \emph{intrinsic node map shape} 
and an \emph{intrinsic edge map shape} and 
These intrinsic shape and dimensions are used to use 
numpy arrays with the best fitting dimension and shape.
For a 2D grid graph the node map should also be a 2D dimensional array.
because in this way an usual image can be used as an node map.
To access the  intrinsic shapes of node and edge maps 
we use a small trait class with default implementations
for unknown graphs.
The default implementation is given below:

\begin{minipage}{\textwidth}\vspace{-0.75cm}\begin{lstlisting}[language=c++]
template<class GRAPH>
class IntrinsicGraphShape{
private:
    typedef GRAPH Graph;
    typedef typename vigra::MultiArray<1,int>::difference_type DiffType1d;
    typedef typename Graph::index_type  index_type;
public:
    typedef typename Graph::Node Node ;
    typedef typename Graph::Edge Edge ;
    typedef typename  Graph::Arc  Arc ;

    typedef DiffType1d IntrinsicNodeMapShape;
    typedef DiffType1d IntrinsicEdgeMapShape;
    typedef DiffType1d  IntrinsicArcMapShape;

    static IntrinsicNodeMapShape intrinsicNodeMapShape(const Graph & g){
        return IntrinsicNodeMapShape(g.maxNodeId()+1);
    }
    static IntrinsicEdgeMapShape intrinsicEdgeMapShape(const Graph & g){
        return IntrinsicEdgeMapShape(g.maxEdgeId()+1);
    }
    static IntrinsicArcMapShape intrinsicArcMapShape(const Graph & g){
        return  IntrinsicArcMapShape(g.maxArcId()+1);
    }

    static const unsigned int IntrinsicNodeMapDimension=1;
    static const unsigned int IntrinsicEdgeMapDimension=1;
    static const unsigned int IntrinsicArceMapDimension=1;
};
\end{lstlisting}\end{minipage}\vspace{0.5cm}



\subsubsection{Numpy Arrays To LEMON Maps}


On the C++ side, numpy arrays are stored in MultiArrayViews.
Since the API of MultiArrayViews \cite{software_vigra_multiarray_api} does
not implemented the API  of LEMON graph maps (e.g. node-maps, edge-maps and arc-maps), 
a thin wrapper is used to convert the arrays to LEMON conform maps.
These wrappers can be accessed via \lstinline{PyNodeMapTraits<Graph,T>::Array} and 
\lstinline{PyNodeMapTraits<Graph,T>::Map}. 


In the following we will give a brief example how to pass numpy arrays to C++
and convert them to LEMON conform graph maps.

Assuming we need a function which does something with node features as normalization.
On the python side we want to have the following signature:

\lstinline{result=vigra.graphs.normNodeFeat(graph,nodeFeatures=nodeFeatures,out=None)}.

The function should work on single band node features and multi band node features
(an arbitrary number of channels, but the same for all nodes).
There should be a single C++ function which can be used to export
\lstinline{normNodeFeat} for any graph within VIGRA's graph API. 




\begin{minipage}{\textwidth}

To archive this we propose the following design:
On the C++ side we use a function  which
is has two templates , one for the graph, and one for the value type.
A prototypical implementation is given below.

\begin{minipage}{\textwidth}\vspace{-0.75cm}\begin{lstlisting}[language=c++]
template<class Graph,class T>
NumpyAnyArray normNodeFeat(
    const Graph & g,
    const typename PyNodeMapTraits<Graph,T >::Array & nodeFeaturesInArray,
    typename PyNodeMapTraits<Graph,T>::Array  outArray 
){
    // reshape out 
    // - last argument (outArray) will be reshaped if empty,
    // - and #channels is taken from second argument (nodeFeaturesIn) 
    reshapeNodeMapIfEmpty(g,nodeFeaturesIn,outArray);

    // numpy arrays => lemon maps 
    // featuresInMap and outMap fulfill
    // the concept of LEMON NODE MAPS
    typename PyNodeMapTraits<Graph,T >::Map featuresInMap(g,nodeFeaturesInArray);
    typename PyNodeMapTraits<Graph,T >::Map outMap(g,outArray);


    /* call code using LEMON API*/

    // return out as numpy array
    return outArray;
}
\end{lstlisting}\end{minipage}\vspace{0.5cm}

The template \lstinline{T} can be instantiated with scalars as \lstinline{float}, an explicit single band scalar as \lstinline{Singleband<float>}
or a multi band type as \lstinline{Multiband<float>}.

\lstinline{PyNodeMapTraits<Graph,T >::Array} will select the correct \lstinline{vigra::NumpyArray<DIM,VALUE_TYPE>} w.r.t.
the templates \lstinline{Graph} and \lstinline{T}.

The output array will be reshaped with the corrected number of channels by calling \lstinline{reshapeNodeMapIfEmpty}.
Equivalent functions exist for edge maps.

\lstinline{PyNodeMapTraits<Graph,T >::Map} is the corresponding LEMON conform  node map 
which is a cheap view to an numpy array. 
\end{minipage}


\begin{minipage}{\textwidth}
To make the function above available in python
we need to write wrapper code with \lstinline{boost::python}.
We will export the function for single band floats and multi band floats.

\begin{minipage}{\textwidth}\vspace{-0.75cm}\begin{lstlisting}[language=c++]
// single-band float32
boost::python::def(
    "normNodeFeat",
     &normNodeFeat<Graph,float>,
    (
        boost::python::arg("graph"),
        boost::python::arg("nodeFeatures"),
        boost::python::arg("out")=boost::python::object() // None
    )
);

// multi-band float32
boost::python::def(
    "normNodeFeat",
     &normNodeFeat<Graph,Multiband<float> >,
    (
        boost::python::arg("graph"),
        boost::python::arg("nodeFeatures"),
        boost::python::arg("out")=boost::python::object() // None
    )
);
\end{lstlisting}\end{minipage}\vspace{0.5cm}

\end{minipage}


On the python side we can call the function as desired.
The array which stores the result of this
function can be preallocated and passed explicitly.

\begin{minipage}{\textwidth}\vspace{-0.75cm}\begin{lstlisting}[language=Python]
// with automatically allocated outFeatures
outFeatures = vigra.graphs.normNodeFeat(graph,nodeFeatures)

// with explicitly given outFeatures
// - here we assume single band features
outFeatures2 = vigra.graphs.graphMap(graph,item='node',dtype=np.float32)
outFeatures2 = vigra.graphs.normNodeFeat(graph,nodeFeatures,out=outFeatures2)


// with multiband node features 
// - here we assume multi band features
//   with 3 channels per node
outFeatures2 = vigra.graphs.graphMap(graph,item='node',dtype=np.float32,channels=3)
outFeatures2 = vigra.graphs.normNodeFeat(graph,nodeFeatures,out=outFeatures2)
\end{lstlisting}\end{minipage}\vspace{0.5cm}




\subsection{Graph Hierarchy}
    
Explain the very nice workflow 