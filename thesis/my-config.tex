%%%%%%%%%%%%%%%%%%%%%%%%%%%%%%%%%%%%
% from cgc paper
%%%%%%%%%%%%%%%%%%%%%%%%%%%%%%%%%%%%
\usepackage{ifthen}
\usepackage[ruled,vlined,linesnumbered]{algorithm2e}

\newcommand{\y}{\ensuremath{\boldsymbol y}}
\newcommand{\Labels}{\ensuremath{\vec{L}}}
\newcommand{\CUT}{\ensuremath{\text{CUT}}}
\newcommand{\w}{\ensuremath{w}}
\newcommand{\MC}{\ensuremath{\text{MC}}}
\newcommand{\CC}{\ensuremath{\text{CC}}}
\newcommand{\ol}[1]{\overline{#1}}
\newcommand{\ExpAndExplore}{Expand \& Explore}

\renewcommand{\vec}[1]{\mathbf{#1}}


\DeclareMathOperator*{\argmin}{\arg\!\min}


\usepackage{todonotes}



%%%%%%%%%%%%%%%%%%%%%%%%%%%%%%%%%%%%
% TIKZ
%%%%%%%%%%%%%%%%%%%%%%%%%%%%%%%%%%%%
\usepackage{tikz,times}
\usetikzlibrary{shapes,arrows,chains,mindmap,backgrounds,positioning,calc}


\usepackage{amsmath,bm,times}
\newcommand{\mx}[1]{\mathbf{\bm{#1}}} % Matrix command
\newcommand{\vc}[1]{\mathbf{\bm{#1}}} % Vector command


\usepackage{tikz,times}
\usepackage[paperwidth=25cm,paperheight=22cm,left=1cm,top=1cm]{geometry}


\usetikzlibrary{shadows,arrows}
% Define the layers to draw the diagram
\pgfdeclarelayer{background}
\pgfdeclarelayer{foreground}
\pgfsetlayers{background,main,foreground}
 
% Define block styles  
\tikzstyle{materia}=[draw, fill=blue!20, text width=6.0em, text centered,
  minimum height=1.5em,drop shadow]
\tikzstyle{practica} = [materia, text width=8em, minimum width=10em,
  minimum height=3em, rounded corners, drop shadow]
\tikzstyle{texto} = [above, text width=6em, text centered]
\tikzstyle{linepart} = [draw, thick, color=black!50, -latex', dashed]
\tikzstyle{line} = [draw, thick, color=black!50, -latex']
\tikzstyle{ur}=[draw, text centered, minimum height=0.01em]
 
% Define distances for bordering
\newcommand{\blockdist}{1.3}
\newcommand{\edgedist}{1.5}

\newcommand{\practica}[2]{node (p#1) [practica]
  {Pr\'actica #1\\{\scriptsize\textit{#2}}}}

% Draw background
\newcommand{\background}[5]{%
  \begin{pgfonlayer}{background}
    % Left-top corner of the background rectangle
    \path (#1.west |- #2.north)+(-0.5,0.5) node (a1) {};
    % Right-bottom corner of the background rectanle
    \path (#3.east |- #4.south)+(+0.5,-0.25) node (a2) {};
    % Draw the background
    \path[fill=yellow!20,rounded corners, draw=black!50, dashed]
      (a1) rectangle (a2);
    \path (a1.east |- a1.south)+(0.8,-0.3) node (u1)[texto]
      {\scriptsize\textit{Unidad #5}};
  \end{pgfonlayer}}

\newcommand{\transreceptor}[3]{%
  \path [linepart] (#1.east) -- node [above]
    {\scriptsize Transreceptor #2} (#3);}


%%%%%%%%%%%%%%%%%%%%%%%%%%
%                        %
% COLORS                 %   
%                        %
%%%%%%%%%%%%%%%%%%%%%%%%%%
\definecolor{code_green}{rgb}{0,0.6,0}
\definecolor{code_gray}{rgb}{0.5,0.5,0.5}
\definecolor{code_mauve}{rgb}{0.58,0,0.82}


%%%%%%%%%%%%%%%%%%%%%%%%%%
%                        %
% SYNTAX HIGHLIGHT       %   
%                        %
%%%%%%%%%%%%%%%%%%%%%%%%%%
\usepackage{listings}
\lstset{ %
  backgroundcolor=\color{white},   % choose the background color
  basicstyle=\footnotesize,        % size of fonts used for the code
  breaklines=true,                 % automatic line breaking only at whitespace
  captionpos=b,                    % sets the caption-position to bottom
  commentstyle=\color{code_green},    % comment style
  escapeinside={\%*}{*)},          % if you want to add LaTeX within your code
  keywordstyle=\color{blue},       % keyword style
  stringstyle=\color{mymauve},     % string literal style
}


