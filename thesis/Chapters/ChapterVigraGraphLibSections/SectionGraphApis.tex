% !TEX root = ../../main.tex

\section{Graph APIs}\label{sec:graph_apis}



We strongly belief it is beneficial to use an existing graph API,
instead of inventing an own graph API for VIGRA.
Coming up with a new API is not straight forward
since one might to think of any future use case.
In addition, users became accustomed with existing API as 
the LEMON \citep{lemon_lib} or BOOST graph API \citep{ boost_bgl}.
Existing algorithms might be reused, as \cite{straehle_2011_miccai} which is implemented 
within LEMON's API.






\subsection{LEMON Graph API}\label{sec:lemon_graph_apis}
    LEMON \citep{ lemon_lib} 
    stand for  ``Library for Efficient Modeling and Optimization in Networks.''.
    It is an open source C++ library with algorithms and data structures 
    related to directed and undirected graphs.
    The extensive usage of templates makes this library very flexible.
    While LEMON provides a huge set of graph algorithms,
    we are mostly interested in the graph API itself.
    In the following, we will give a brief overview of lemons graph 
    API and the related concepts.
    Explaining the complete  API in detail
    is beyond the scope of this thesis.
    Interested readers are referred to work of \citet{lemon_lib}.
    We will only discuss the API for undirected graphs since any
    graph algorithm we implemented within this thesis
    will work on undirected graphs exclusively.

\paragraph{Graph Items :}
    Any undirected graph class fulfilling the LEMON API needs to define 
    the following \emph{descriptor} types to represent the graph items:
    \begin{inparaenum}[(i)]
    \item \lstinline{Graph::Node},
    \item \lstinline{Graph::Edge} and
    \item \lstinline{Graph::Arc}.
    \end{inparaenum}
    These \emph{descriptor} should be cheap types which can be copied
    and passed with almost no overhead.
    In addition, each descriptor has an unique id
    \footnote{ unique id w.r.t. the item type. 
    Therefore  multiple  nodes cannot have the same id.
    The same holds line for edges and arcs.
    But there might be a node and edge which have the same id}.
    These ids can be accessed via \lstinline{Graph::id(Node)}, \lstinline{Graph::id(Edge)} and \lstinline{Graph::id(Arc)}.
    These ids can not only be dense but also sparse, which is very
    important for an efficient handling of grid graph edge ids (see \cref{sec:graphs_grid_graph}).


\phantomsection
\label{par:lemon_iterators}
\paragraph{Iterators :}
    Within LEMON a very convenient mechanism is used to iterate over
    nodes, edges and arcs.
    A special constant \lstinline{INVALID} is used to determine if 
    an iterator reached the end.

    \begin{lstlisting}[language=c++]
    // iterate over nodes
    for(Graph::NodeIt v(g); v!= lemon::INVALID; ++v){/*...*/}

    // iterate over edges
    for(Graph::EdgeIt e(g); e!= lemon::INVALID; ++e){/*...*/}

    // iterate over arcs
    for(Graph::ArcIt a(g); a!= lemon::INVALID; ++a){/*...*/}

    // use arcs to iterate over neighbor nodes
    for(Graph::OutArcIt a(g,n); a!= lemon::INVALID; ++a){
        const Node neighborNode = g.target(a);
        /*...*/
    }
    \end{lstlisting}

    Any iterator is convertible to the corresponding item which
    is iterated without using \lstinline{operator*()}.

\paragraph{Map Concept :}
    The separation between the graph, and data which is related to
    the graph is crucial.
    Different algorithms which operate on the same graph will
    need different data attached to the edges, nodes and arcs. 

    In LEMON, graph classes store only the structure of the graph itself.
    All addition data for nodes, edges and arcs is stored 
    in \emph{maps}.
    The API for graph maps in lemon is very small and easy to implement.
    In fact, only a constructor and \lstinline{operator[](...)} needs
    to be implemented (see \cref{fig:uml_lemon_graph_concepts} for an UML class diagram).



    \begin{figure}[H]
    \begin{center}
        \begin{tikzpicture}[scale=0.55,transform shape]
            \begin{umlpackage}{LEMON API}
                    \umlclass[x=-15.5,y=0]{UndirectedGraphConcept}
                {
                }
                {
                    // Typedefs \usp \\
                    + Node \usp \\
                    + Edge \usp \\
                    + Arc  \usp \\
                    + NodeIt \usp \\
                    + EdgeIt \usp \\
                    + ArcIt  \usp \\
                    + IncEdgeIt \usp \\
                    + InArcIt   \usp \\
                    + OutArcIt  \\

                    // Nested Classes  \\
                    + NodeMap \nestedtemp{ValueType}   \\
                    +  EdgeMap \nestedtemp{ValueType}   \\
                    +  ArcMap \nestedtemp{ValueType}   \\

                    // Member Functions
                    + u(edge : Edge) : Node \usp \\
                    + v(edge : Edge) : Node \usp \\
                    + source(arc : Arc) : Node \usp \\
                    + target(arc : Arc) : Node \usp \\
                    + id(node : Node) : int \usp \\
                    + id(edge : Edge) : int \usp \\
                    + id(arc  : Arc)  : int \usp \\
                    + nodeFromId(id : int) : Node \usp \\
                    + edgeFromId(id : int) : Edge \usp \\
                    + arcFromId(id  : int) : Arc  \usp \\
                    + maxNodeId() : int \usp \\
                    + maxEdgeId() : int \usp \\
                    + maxArcId()  : int \usp \\
                    + direction(arc : Arc) : bool \usp \\
                    + direct(edge : Edge, naturalDirection : bool) :Arc \usp \\
                    + direct(edge : Edge, node : Node) :Arc \usp \\
                    + oppositeArc(arc : Arc) : Node \usp \\
                    + oppositeNode(node : Node, edge : Edge) : Node \usp \\
                    + baseNode(iter : IncEdgeIt) : Node \usp \\
                    + runningNode(iter : IncEdgeIt) : Node \usp \\
                    + baseNode(iter : OutArcIt) : Node \usp \\
                    + runningNode(iter : OutArcIt) : Node \usp \\
                    + baseNode(iter : InArcIt) : Node \usp \\
                    + runningNode(iter : InArcIt) : Node 
                }
                %\begin{umlpackage}{Graph Item Concept}
                    \umlclass[x=-6.5,y=7]{Node}
                    {
                    }
                    {   
                        + Node() \usp \\ 
                        + Node(node : Node) \usp \\ 
                        + Node(invalid: Invalid)  \usp \\ 
                        + operator == (node : Node) : bool \usp \\ 
                        + operator != (node : Node) : bool \usp \\ 
                        + operator <  (node : Node) : bool \usp \\ 
                        \quad
                    } 
                    \umlclass[x=0,y=7]{Edge}
                    {
                    }
                    {  
                        + Edge() \usp \\ 
                        + Edge(edge : Edge) \usp \\ 
                        + Edge(invalid: Invalid)  \usp \\ 
                        + operator == (edge : Edge) : bool \usp \\ 
                        + operator != (edge : Edge) : bool \usp \\ 
                        + operator <  (edge : Edge) : bool \usp \\
                        \quad
                    } 
                    \umlclass[x=6.5,y=7]{Arc}
                    {
                    }
                    {   
                        + Arc() \usp \\ 
                        + Arc(arc : Arc) \usp \\ 
                        + Arc(invalid: Invalid)  \usp \\ 
                        + operator == (arc : Arc) : bool \usp \\ 
                        + operator != (arc : Arc) : bool \usp \\ 
                        + operator <  (arc : Arc) : bool \usp \\ 
                        + operator Edge () : Edge 
                    } 
                %\end{umlpackage}


                %\begin{umlpackage}{Graph Item Iterator Concept}
                    \umlclass[x=-7,y=0]{NodeIt}
                    {
                    }
                    {   
                        + NodeIt() \usp \\ 
                        + NodeIt(iter : NodeIt) \usp \\ 
                        + NodeIt(invalid: Invalid)  \usp \\ 
                        + NodeIt(g: Graph)  \usp \\ 
                        + NodeIt(g: Graph, node: Node)  \usp \\ 
                        + operator++(): NodeIt \\ 
                    }
                    \umlclass[x=0,y=0]{EdgeIt}
                    {
                    }
                    {   
                        + EdgeIt() \usp \\ 
                        + EdgeIt(iter : EdgeIt) \usp \\ 
                        + EdgeIt(invalid: Invalid)  \usp \\ 
                        + EdgeIt(g: Graph)  \usp \\ 
                        + EdgeIt(g: Graph, edge: Edge)  \usp \\ 
                        + operator++(): EdgeIt \\ 
                    }
                    \umlclass[x=7,y=0]{ArcIt}
                    {
                    }
                    {   
                        + ArcIt() \usp \\ 
                        + ArcIt(iter : ArcIt) \usp \\ 
                        + ArcIt(invalid: Invalid)  \usp \\ 
                        + ArcIt(g: Graph)  \usp \\ 
                        + ArcIt(g: Graph, arc: Arc)  \usp \\ 
                        + operator++(): ArcIt \\ 
                    } 
                %\end{umlpackage}


                %\begin{umlpackage}{Graph Neighborhood Iterator Concept}
                    \umlclass[x=-7.5,y=-7]{IncEdgeIt}
                    {
                    }
                    {   
                        + IncEdgeIt() \usp \\ 
                        + IncEdgeIt(iter : IncEdgeIt) \usp \\ 
                        + IncEdgeIt(invalid: Invalid)  \usp \\ 
                        + IncEdgeIt(g: Graph, node: Node)  \usp \\ 
                        + IncEdgeIt(g: Graph, node: Node, edge : Edge)  \usp \\ 
                        + operator++(): IncEdgeIt \\ 
                    }
                    \umlclass[x=0,y=-7]{InArcIt}
                    {
                    }
                    {   
                        + InArcIt() \usp \\ 
                        + InArcIt(iter : InArcIt) \usp \\ 
                        + InArcIt(invalid: Invalid)  \usp \\ 
                        + InArcIt(g: Graph, node: Node)  \usp \\ 
                        + InArcIt(g: Graph, node: Node, arc : Arc)  \usp \\ 
                        + operator++(): InArcIt \\ 
                    }
                    \umlclass[x=7.5,y=-7]{OutArcIt}
                    {
                    }
                    {   
                        + OutArcIt() \usp \\ 
                        + OutArcIt(iter : OutArcIt) \usp \\ 
                        + OutArcIt(invalid: Invalid)  \usp \\ 
                        + OutArcIt(g: Graph, node: Node)  \usp \\ 
                        + OutArcIt(g: Graph, node: Node, arc : Arc)  \usp \\ 
                        + operator++(): OutArcIt \\ 
                    }
                    \umlclass[x=-13,y=-16]{NodeMap}
                    {
                    }
                    {   
                        // Typedefs \\
                        + Value \\
                        + ConstReference \\
                        + Reference \\
                        + Key // same as Node for NodeMap \\
                        \\// Members \\
                        +NodeMap(graph : Graph ) \\
                        + operator[](edge : Key) : Reference \\
                        + operator[](edge : Key) : ConstReference \\
                    } 
                    \umlclass[x=-6,y=-16]{EdgeMap}
                    {
                    }
                    {   
                        // Typedefs \\
                        + Value \\
                        + ConstReference \\
                        + Reference \\
                        + Key // same as Edge for EdgeMap \\
                        \\// Members \\
                        +EdgeMap(graph : Graph ) \\
                        + operator[](edge : Key) : Reference \\
                        + operator[](edge : Key) : ConstReference \\
                    } 
                    \umlclass[x=1,y=-16]{ArcMap}
                    {
                    }
                    {   
                        // Typedefs \\
                        + Value \\
                        + ConstReference \\
                        + Reference \\
                        + Key // same as Arc for ArcMap \\
                        \\// Members \\
                        +ArcMap(graph : Graph ) \\
                        + operator[](arc : Key) : Reference \\
                        + operator[](arc : Key) : ConstReference \\
                    } 

                %\end{umlpackage}


                \umlinherit[]{NodeIt}{Node}
                \umlinherit[]{EdgeIt}{Edge}
                \umlinherit[]{ArcIt}{Arc}

                \umlinherit[]{IncEdgeIt}{Edge}
                \umlinherit[]{InArcIt}{Arc}
                \umlinherit[]{OutArcIt}{Arc}

            \end{umlpackage}
        \end{tikzpicture}
    \end{center}
    \caption{
        UML class diagram of most important LEMON concepts.
        UndirectedGraphConcept shows LEMON's API for undirected 
        graphs. Any graph implemented within LEMON's API 
        needs to implement all methods and typedefs showed 
        in the UML diagram.
        To implement a graph class within the API one needs 
        to implement descriptor classes for nodes, edges and arcs.
        These descriptor classes have almost no API.
        The graph itself is responsible to deal with them.
        The descriptors are usually implemented as cheap classes,
        and are used to \emph{describe} a node, and do not implement
        functionality for nodes.
        The functionality comes from the graph class itself.
        Iterators in LEMON are somehow different from usual 
        iterators, instead of an end iterator, lemon used a special
        class \lstinline{lemon::INVALID}, and all iterators
        need to comparable with \lstinline{lemon::INVALID} (see \cref{par:lemon_iterators}).
    }\label{fig:uml_lemon_graph_concepts}
    \end{figure}






    %\begin{minipage}{\textwidth}
        Any graph has a templated default implementations for graph maps.
        An edge map with \lstinline{float} as value type, which might
        be used as an edge indicator/weight, can be 
        used in the following way.

        \begin{lstlisting}[language=c++]
        // edge map (for data as edge weights)
        Graph::EdgeMap<float> edgeMap(g); 
        for(Graph::EdgeIt e(g); e!= lemon::INVALID; ++e){
            const float val = edgeMap[*e];  // read
            edgeMap[*e] = std::exp(-1.0*a); // write
        }
        \end{lstlisting}
    %\end{minipage}

    %\begin{minipage}{\textwidth}
        A node map with  \lstinline{unsigned int} as value type,
        which could encode a labeling for a graph 
        can be accessed as shown below.

        \begin{lstlisting}[language=c++]
        // node map (for node related data as node labelings )
        Graph::NodeMap<usigned int> nodeMap(g);
        for(Graph::NodeIt v(g); v!= lemon::INVALID; ++v){
            const unsigned int val = nodeMap[*v]; // read
            nodeMap[*v] = val+1;                  // write
        }
        \end{lstlisting}
    %\end{minipage}


    % %\begin{minipage}{\textwidth}
    %     Implicit read only graph maps can be implemented very easy and 

    %     \begin{minipage}{\textwidth}\vspace{-0.75cm}\begin{lstlisting}[language=c++]
    %     template<class Graph>
    %     class ImplicitEdgeMap {
    %     public:
    %         typedef typename Graph::Edge Key;
    %         typedef double Value;
    %         ImplicitEdgeMap(const Graph & graph){
    %             /*
    %                 constructor code here
    %             */
    %         }
    %         Value operator[](const Key & edge) const { 
    %             Value a;
    %             /*
    %                 compute a value for the edge
    %                 implicity here
    %             */
    %             return a;
    %         }
    %     };
    %     \end{lstlisting}\end{minipage}\vspace{0.5cm}
    % %\end{minipage}

\subsection{BOOST Graph API}\label{sec:boost_graph_apis}
The BOOST Graph Library (BGL)  \citep{boost_bgl} is set of data structures and 
algorithms for graph related computations.
Since all  data-structures and algorithms presented within this thesis  are implemented within the LEMON graph interface, 
the BGL graph API will only be described briefly.

The main difference between the LEMON graph API and the BGL,
is the massive usage of trait classes and free functions within the BGL.
While LEMON classes use typedefs and member functions, 
the BGL uses trait classes and free functions which need 
to be implemented or specialized for a particular graph.

Node and edge descriptors are used in the same way as in LEMON's
graph API \footnote{within BGL these descriptors are called \lstinline{vertex_descriptor} and 
\lstinline{node_descriptor}}.
There are no restrictions how these descriptors can be implemented.
For most graph algorithms, the BGL expects a list of dense integers for
node and edge ids.
With this requirement, any flexibility of custom edge and node descriptors is lost.
We belief that this is a poor design choice, since sparse ids
are crucial for most graphs (see \cref{sec:impl_graphs}).




\subsection{VIGRA Graph API}

While the grid graph is implemented within the BGL and LEMON graph API,
all algorithms and data-structures implemented within this thesis will
use the LEMON graph API instead of the BGL 
for the following reasons:
\begin{inparaenum}[(i)]
\item Algorithms using the BGL API expect not only the graph, but also 
a list of dense integers, while LEMON allows for sparse ids which 
are part of the graph itself.
\item LEMON seems to be well maintained and under active development,
while for the BGL there is no current development at all.
\item The documentation of LEMON is excellent, while the BGL is
documented in a very confusing way.
  
\end{inparaenum}

